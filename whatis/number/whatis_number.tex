\documentclass{article}
\usepackage{../mysty}
\addbibresource{../bibliography.bib}

\title{What is a Number?}


\begin{document}
\maketitle

\section{Introduction}\label{sec:intro}
The author is unaware of an universally accepted definition of ``number'' that unifies all of the standard notions. The tempting definition of \textit{a set of mathematical objects with which one can do arithmetic with} seems too large (likely including all fields) and simultaneously too restricting (likely excluding the natural numbers), depending on what one means by ``arithmetic.'' Instead, we will define the standard sets: the natural numbers ($\N$), the integers ($\Z$), the rational numbers ($\Q$), the real numbers ($\R$), and the complex numbers ($\C$).


\section{The Natural Numbers}\label{sec:naturals}
In order to define the natural numbers, we must have a notion of what a \textit{set} is.

\defn[set]{
	A set is a (possibly empty) collection of mathematical or abstract objects.
}

This definition is completely unsatisfactory in general, but it will serve our purposes for now. \cite{HalmosNST} We will define the natural numbers by the Peano axioms.

\ax{
	Let $\newsymbol{N}{\N}{Natural Numbers}$ denote the set of all natural numbers. We take (axiomatically) that:
	\begin{enumerate}[i)]
		\item $1\in\N$
		\item If $a\in\N$, then $a$ has a \textit{successor} $S(a)$ and $S(a)\in\N$
		\item 
	\end{enumerate}
}




\section*{List of Symbols}
\PrintListOfSymbols

\printbibliography

\end{document}